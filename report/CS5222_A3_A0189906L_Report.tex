\documentclass[multicol,date,tikzlibs,minted,ieeebib,bibtex]{epreport}

\usepackage[]{tabularray}

\module{CS5222 Advanced Computer Architecture}
\title{Project 3: Prefetcher Implementation}
\author{Sharadh Rajaraman}
\matricno{A0189906L}

\addbibresource{CS5222-Project-3.bib}

\newacronym{pc}{PC}{program counter}
\newacronym{ghb-pc-dc}{GHB PC/DC}{global history buffer with \gls{pc} localisation and delta correlation}
\newacronym{it}{IT}{Index Table}
\newacronym{ghb}{GHB}{Global History Buffer}
\newacronym{fifo}{FIFO}{first-in, first-out}
\newacronym{lru}{LRU}{least recently used}

\begin{document}
\maketitle

\begin{multicols}{2}
	\section{Introduction}
	\subsection{Data Prefetching}
	Prefetching is to data, what branch prediction is to text: both aim to alleviate stalled cycles as a result of fetching text or data lines from main memory, by guessing future accesses given a past history.
	In this project, the \gls{ghb-pc-dc} prefetcher has been implemented.
	\subsection{\texorpdfstring{\gls{ghb-pc-dc} Overview}{GHB PC/DC Overview}}
	\citeauthor{nesbitDataCachePrefetching2004} provide a detailed overview of \gls{ghb-pc-dc} prefetching\cite{nesbitDataCachePrefetching2004}; a very quick summary is detailed as follows.

	The prefetcher consists of two data structures: an \gls{it}, which is a hash-table mapping some localisation key to pointers into a \gls{ghb}, which in turn is a fixed-size \gls{fifo} queue (also called a circular or ring buffer), containing pairs of miss addresses, and pointers to other elements in the \gls{ghb}.
	For \gls{ghb-pc-dc}, the indexing key is the \gls{pc}.

	\subsection{\texorpdfstring{\gls{ghb-pc-dc} Operation}{GHB PC/DC Operation}}\label{subsec:ghb-pc-dc-oper}
	The \gls{ghb-pc-dc} prefetcher's algorithm is detailed (albeit rather sparsely) in \citetitle{falsafiPrimerHardwarePrefetching2014}\cite{falsafiPrimerHardwarePrefetching2014}.
	A detailed algorithm is listed below, including edge cases handled by neither \citeauthor{nesbitDataCachePrefetching2004} nor \citeauthor{falsafiPrimerHardwarePrefetching2014}.

	\begin{enumerate}
		\item Upon a cache \emph{miss}, the index table is indexed with the \gls{pc}.
		      \begin{enumerate}
			      \item If there is a hit, then go to 2.
			      \item If there is no hit, or the location dereferenced by the corresponding pointer has been evicted from the \gls{ghb}, an entry is created, and no prefetching is performed.
		      \end{enumerate}
		\item The miss address is saved onto the stack, and the corresponding pointer is inserted into the head of the \gls{ghb}.
		\item Given the previous pointer from the \gls{it} hit, the `next' pointer of the above new miss address is set to the previous pointer, thus building a linked list.
		\item The pointer is dereferenced to the GHB, and the linked list is followed, adding every miss address found, to an array of addresses accessed by the same \gls{pc}.
		\item Given the nature of the linked list, this array is in reverse-chronological order, with the latest miss address at the beginning of the array, and the earliest miss address at the end. Therefore, this array is reversed.
		\item Using this reversed array, the \emph{last three} elements in said array are used to build the \emph{latest} delta pair, i.e. the two differences between said last three elements.
		\item A delta stream is built given the differences between each element in the array, too.
		\item The delta stream is searched to find the \emph{last} occurrence of the latest delta pair (excluding that final occurrence itself).
		\item From \emph{this} occurrence of the delta pair, up to the defined prefetch degree number of deltas are collected, searching \emph{forwards} in the delta stream.
		\item Finally, the latest miss address is added to each delta, and these are the addresses to be prefetched.
	\end{enumerate}

	\section{Implementation}
	The prefetcher was implemented in \Cpp{20}, and the standard library was put to great use while building the delta stream.

	\subsection{Data Structures}
	To begin with, to simulate hardware limitations, the \gls{it} was built using an implementation of a \gls{lru} hash table, with a fixed size.
	The \gls{ghb} was implemented using \mintinline{c++}{boost::circular_buffer}, in the Boost library.
	Both data structures have fixed, equal capacities.

	\subsection{Pointer and Memory Management}
	The existence of pointers and the dynamic destruction of data by the \gls{ghb} poses a potential memory management issue: multiple pointers to the same data would mean that as data is evicted by the \gls{ghb}, the pointers in the \gls{it} and \emph{within} the \gls{ghb} are not invalidated correctly.

	To solve this, the GHB is set to \emph{own} the pointers.
	As such, it contains \mintinline{c++}{std::shared_ptr<Node>}, with the following definition of \mintinline{c++}{Node}:
	\begin{minted}{c++}
	struct Node {
		// ... constructor ...
		AddressType miss_address;
		std::weak_ptr<Node> next;
	}
	\end{minted}
	The \emph{entries} in the \gls{it} are of type \mintinline{c++}{std::weak_ptr<Node>}, too.
	This allows the graceful destruction of the \mintinline{c++}{Node}s as they are evicted from the \gls{ghb}.
	Any \mintinline{c++}{std::weak_ptr}s that dereference to said \mintinline{c++}{Node} may be queried for validity by way of \mintinline{c++}{std::weak_ptr<T>::expired()}.

	It is worth noting that \mintinline{c++}{std::adjacent_difference} and \mintinline{c++}{std::find_end} were critical methods from the \Cpp{} standard library to build and verify the delta streams mentioned in \cref{subsec:ghb-pc-dc-oper}.

	\subsection{Implementation Discussion}
	The implementation was relatively straightforward, and completed with \texttt{CLion 2022.1}, with \texttt{clang++} version 13 and \texttt{gdb} version 11.2.
	Initially, a templated implementation was sought; however, to ease debugging and development, this was quickly abandoned in favour of hardcoding types in the \gls{it} and \gls{ghb} implementations.

	It was very pleasant to see the linked lists being built in the \gls{ghb}, as verification that the implementation was indeed correct.

	\subsection{Parameters}
	The parameters provided to the \gls{ghb-pc-dc} prefetcher are the same as the literature: 256 entries in the \gls{it} and \gls{ghb} each, with a maximum prefetch degree of 4.

	\subsection{Result Collection}
	The \texttt{stdout} from each program run was written to a correspondingly-named file in the \texttt{output/} directory.
	Then, the \texttt{make\_csv.sh} script compiled the final IPC output from each simulator run into a \texttt{.csv} file for further mangling and data processing in a spreadsheet/chart generation program.

	\section{Results and Discussion}
	The results of the implementation are compared with the baseline no-prefetcher simulator i.e. \texttt{skeleton.cc}.
	The percentage improvements of each implementation is detailed in \cref{fig:prefetcher-perf}.
\end{multicols}
\begin{figure}[htbp]
	\centering
	\begin{tikzpicture}
	\pgfplotsset{
		discard if not symbolic/.style 2 args={
			% suppress LOG messages about the filtered points
			filter discard warning=false,
			x filter/.append code={
				\edef\tempa{\thisrow{#1}}
				\edef\tempb{#2}
				\ifx\tempa\tempb
				\else
					\def\pgfmathresult{NaN}
				\fi
			},
		},
	},
	\begin{axis}[
		width=15cm,
		height=5cm,
		ymajorgrids,
		ytick={-20,0,20,40,60,80,100},
		ybar=0pt,
		ylabel={IPC improvement (\%)},
		symbolic x coords={gcc,GemsFDTD,lbm,leslie3d,libquantum,mcf,milc,omnetpp},
		xtick style={draw=none},
		ytick style={draw=none},
		bar width=0.25cm,
		legend style={draw=none,column sep=0.5cm},
		legend cell align=right,
		legend pos=outer north east,
		ticklabel style = {font=\footnotesize},
	]
		\foreach \i in {ampm-lite,ghb-pc-dc,ip-stride,next-line,stream} {
			\addplot+ [discard if not symbolic={trace}{\i}] table [
					col sep=comma,
					x=prefetcher,
					y=ipc-per-diff,
				] {../percentages.csv};
			\addlegendentryexpanded{\i}
		}
	\end{axis}
\end{tikzpicture}

	\caption{Prefetcher performance}\label{fig:prefetcher-perf}
\end{figure}
\begin{multicols}{2}
	\subsection{Comparisons}
	It was not surprising to see the AMPM prefetcher outperforming \gls{ghb-pc-dc}: this is a \emph{very} recent prefetcher and in benchmarks by the authors (\citeauthor{ishiiAccessMapPattern2011}), performs up to twice as well as \gls{ghb-pc-dc}\cite{ishiiAccessMapPattern2011}

	What \emph{is} surprising, however, is the comparatively poor performance of \gls{ghb-pc-dc} versus IP Stride and even Next Line.
	While debugging, it was noted that \gls{ghb-pc-dc} was not performing as well as it \emph{should} have been.
	A large proportion of access deltas were found to be rather \emph{constant} for many benchmarks, meaning that only \emph{two} addresses were prefetched, rather than the maximum specified by the prefetch degree.

	It is possible that additional heuristics, such as finding the second-latest occurrence of the delta pair, searching for \emph{multiple} delta pairs in the stream and somehow concatenating/averaging these results, or simply excluding some arbitrary initial portion of the linked-list, could lead to improved performance for the \gls{ghb-pc-dc} prefetcher.

	\section{Conclusion}
	In conclusion, a \glsentryfull{ghb-pc-dc} prefetcher was built, tested, and run using the Data Prefetching Championship framework, and these results were compared with example prefetchers provided in the implementation.
\end{multicols}

\printbibliography{}
\end{document}
